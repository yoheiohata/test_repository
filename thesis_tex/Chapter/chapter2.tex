%#!platex main.tex
\chapter{従来法}
%\chapter{ノイズを除去するフィルタの問題点}
 本研究では,特定の周波数ノイズを除去することを目的とする.したがって,本章ではヒルベルト変換前のノイズ除去として考えられる手法と問題点について記す.まず,%
帯域通過ヒルベルト変換器について述べる.次に,伝送零点を有するFIRフィルタについて述べる.最後に,指定した位置に伝送零点を有するFIRフィルタと単純接続した帯域通過ヒルベルト変換器について述べる.
\section{帯域通過ヒルベルト変換器}
一般にノイズを含む信号に対してヒルベルト変換を行うためには,
帯域通過フィルタによりノイズを低減させた後に,ヒルベルト変換器を縦続接続する手法が考えられる.
しかし,帯域通過フィルタとヒルベルト変換器を一つのシステムとして見た場合,
個々のフィルタを別々に用いるため,システム全体の次数が増加し結果として遅延や回路規模の増大につながり好ましくない.
そこでまず,帯域通過フィルタとヒルベルト変換器を合成したフィルタ,
すなわち振幅特性に阻止域を有する帯域通過ヒルベルト変換器を考える.
振幅理想特性は
\begin{equation}\label{ideal_fresp_HT}
D_{\mathrm{BPHT}}(\omega)=\begin{cases}
0\;\;\;&(0 <  \omega  < \omega_1)\\
1\;\;\;&(\omega_1 < \omega < \omega_2)\\
0\;\;\;&(-\omega_2 <  \omega  < \pi)\\
\end{cases}
\end{equation}
と書き表せる.ここで$\omega_1$,$\omega_2$はそれぞれ左側通過域端正規化角周波数,右側通過域端正規化角周波数を表す.また,$N$次のFIRフィルタの周波数応答は
\begin{equation}\label{FIRfresp}
H(\omega)=\sum_{n=0}^N{h(n)e^{-j{\omega}n}}
\end{equation}
と書き表せる.帯域通過フィルタの周波数応答を$H_{\mathrm{BP}}(\omega)$,ヒルベルト変換器の周波数応答を$H_{\mathrm{HT}}(\omega)$とすると,帯域通過ヒルベルト変換器の周波数応答$H_{\mathrm{BPHT}}$は
\begin{eqnarray}\label{BPHTfresp}
  H_{\mathrm{BPHT}}(\omega)=H_{\mathrm{BP}}(\omega)H_{\mathrm{HT}}(\omega)={\sum_{m=0}^{M}{h_{\mathrm{BP}}(m)e^{-j{\omega}m}}}{\sum_{n=0}^{N-M}{h_{\mathrm{HT}}(n)e^{-j{\omega}n}}}
\end{eqnarray}
となる.設計された帯域通過ヒルベルト変換器の振幅特性を図に示す.ここで問題となる点は,阻止域で十分な減衰量を確保するために
フィルタ次数が多く必要となることである.しかし,特定の周波数ノイズのみ減衰させる場合,
阻止域全体で減衰量を確保する必要はなく,特定の周波数近傍で十分な減衰量が確保されればよい.
したがって,阻止域上に伝送零点を有するヒルベルト変換器を設計すればよい.
次節では,指定した位置に伝送零点を有するFIRフィルタと単純接続したヒルベルト変換器について述べる.

\section{伝送零点を有するFIRフィルタと単純接続した帯域通過ヒルベルト変換器}
任意の点$\omega_{i}$に伝送零点を有するFIRフィルタの周波数応答$P(\omega)$は
\begin{equation}
  P(e^{j{\omega}})=\prod_{i=1}^L{(1-2\cos{\omega_{i}}e^{-j{\omega}}+e^{-2j{\omega}})}
\end{equation}
と書き表せる.ここで$L$は伝送零点の個数である.
これより指定した周波数において大きな減衰量を持つフィルタが設計されていることが確認できる.
そのため,特定の周波数ノイズに対して,伝送零点を設置することでノイズの除去が可能となる.
このフィルタに対して帯域通過ヒルベルト変換器を縦続接続する.式\eqref{FIRfresp}より
伝送零点フィルタの周波数応答を$H_{\mathrm{n}}(\omega)$,
式\eqref{BPHTfresp}より帯域通過ヒルベルト変換器の周波数応答$H_{\mathrm{BPHT}}(\omega)$とすると
伝送零点を有するFIRフィルタと単純接続した帯域通過ヒルベルト変換器の周波数応答はアイウエオ
\begin{eqnarray}
  \lefteqn{
H_{\mathrm{nHT}}(\omega)={H_{\mathrm{n}}(\omega)}{H_{\mathrm{ht}}(\omega)}}\quad\nonumber\\
&\!=\!& \prod_{i=1}^{B} \{1\!-\!2\cos\left(\omega_{bi}\right)e^{-j{\omega}}\!+\!e^{-2j{\omega}}\}\sum_{k=0}^{2(M-B)}{h_{\mathrm{ht}}}\left(k\right)e^{-j{\omega}k}
\end{eqnarray}
となる.